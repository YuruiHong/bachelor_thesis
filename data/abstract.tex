% !TeX root = ../thuthesis-example.tex

% 中英文摘要和关键字

\begin{abstract}
  本研究旨在应用机器学习方法,基于中药化合物定量组成数据,对其作用靶点进行预测,并进行实验验证。中药天然产物构成了一个广阔而有用的化学空间,对药物活性分子的研究提供了丰富的资源。然而,中药组成数据的定量面临内在难度、现状、方法等诸多挑战,而中药靶点预测任务除了受定量效果影响外,还存在建模表征预测能力不足的问题。本研究通过收集和标准化《中国药典》收录的中药化合物的定量数据,应用统计学和机器学习方法,使用图模型进行建模,尝试对药方合理性进行评估,以此对处方发现和优化在一定程度上进行辅助和检验。

  % 关键词用“英文逗号”分隔,输出时会自动处理为正确的分隔符
  \thusetup{
    keywords = {中医药天然产物, 药物发现, 中药组成定量, 图模型, 中药靶点预测},
  }
\end{abstract}

\begin{abstract*}
  This study aims to apply machine learning methods to predict the action targets of traditional Chinese medicine (TCM) based on the quantitative composition data of TCM compounds, followed by experimental validation. The natural products of TCM constitute a broad and useful chemical space, offering rich resources for the study of pharmacologically active molecules. However, the quantification of TCM composition faces multiple challenges, including inherent difficulties, current status, and methodologies. Furthermore, the task of predicting TCM targets is impacted by the effectiveness of quantification and the insufficiency of modeling representation and predictive ability. In this study, we collected and standardized quantitative data on TCM compounds listed in the \textit{Chinese Pharmacopoeia}, applied statistical and machine learning methods, and utilized graph models to evaluate the rationality of prescriptions. This approach assists and verifies prescription discovery and optimization to some extent.

  % Use comma as separator when inputting
  \thusetup{
    keywords* = {Traditional Chinese Medicine Natural Products, Drug Discovery, Quantification of TCM Composition, Graph Models, TCM Target Prediction},
  }
\end{abstract*}
