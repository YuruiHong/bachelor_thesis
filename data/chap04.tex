    % !TeX root = ../thuthesis-example.tex

\chapter{总结和展望}

\section{总结}

本研究的工作主要可以归结为数据收集和建模分析两个部分。其中,在数据收集方面,主要对《中国药典》、ccTCM项目原始数据、STRING蛋白互作数据库中和中药化合物有关的基因及其表达关联进行了获取和整理,不仅用于项目本身,也为后续应用其他分析方法提供了基础,如《中国药典》的格式化文本可用于提取中药处方标准各方面的信息;在建模分析方面,我们尝试了对药材配对直接进行作图比较和双样本Kolmogorov-Smirnov非参数检验方法,虽然初步得出了在药材基因交互边数和分数的定量标准下,已知有效处方和随机构造的虚拟处方在总体上有显著差异,从一个侧面反映了处方的有效性可能被网络关系描述,但没有得到直接判断药材配对是否有利于形成有效处方的标准,然后我们尝试了基于图神经网络的方法,通过对药材配对的图结构进行学习,得到了一定的预测效果,对于正负例均衡的样本使用自编码器和随机森林分类器,得到最高0.716的准确率,0.725的AUC和0.72的加权平均F1分数,并验证了使用化合物定量信息可以提高模型训练的有效性,普遍提升预测性能指标,说明在中药处方评估和靶点预测问题中应用定量信息是有帮助的。但是,由于数据量和模型复杂性的限制,预测准确率仍有待进一步提高。

\section{展望}

现阶段,《中国药典》中部分处方定量信息并没有具体提供,或者形式上不够直观可用;ccTCM等中药化合物定量数据库不够完善,对常见的药材也有相当的缺失:这使得我们最终得到的有效正例处方数量相比《中国药典》的收录量较低,限制了模型的学习范围。在未来的工作中,继续对中药处方、药材化合物含量、化合物对基因表达的影响、基因间的交互作用进行定量收集和测定将是有价值的工作,可为中药处方有效性预测的任务提供更多的数据支持,提高预测能力和可靠性。在这些定量信息中,现阶段仍广泛缺失、亟待补充的是化合物对基因表达的影响,或即剂量-效应关系。找到足够规范、统一、可利用的数据表示方法将可能极大地提高从中药处方到基因互作网络关联关系的严谨性。在现有的数据库因为存在大量缺失和较为混乱的量纲而难以使用的状况下,我们希望建立更加标准化的数据记录格式,实现记录数据和缺失数据在同一模型中联合使用的算法,以便于后续的数据整合和分析。

同时,我们也可以尝试更多的深度学习方法或技巧来进行更加符合数据特点的建模。除此之外,适当地引入更多的先验知识,如中药药理学、中药方剂学,以及现代药理学、药效学等领域的知识,将有助于提高模型的可解释性,构成一个更加完整的中药处方有效性预测框架。如果进行充分的调优,达到可接受的预测性能,我们可以尝试使用这个框架来预测新的中药处方的有效性,并在此基础上对处方的基因关联图再进行节点重要性分析,找到其中的关键基因,为中药药效研究提供更多的线索。