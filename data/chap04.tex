    % !TeX root = ../thuthesis-example.tex

\chapter{总结和展望}

\section{总结}

本研究的工作主要可以归结为数据收集和建模分析两个部分。其中,在数据收集方面,主要对《中国药典》、ccTCM项目原始数据、STRING蛋白互作数据库中和中药化合物有关的基因及其表达关联进行了获取和整理,不仅用于项目本身,也为后续应用其他分析方法提供了基础;在建模分析方面,我们尝试了对药材配对直接进行作图比较和双样本Kolmogorov-Smirnov非参数检验方法,虽然初步得出了在药材基因交互边数和分数的定量标准下,已知有效处方和随机构造的虚拟处方在总体上有显著差异,从一个侧面反映了处方的有效性可能被网络关系描述,但没有得到直接判断药材配对是否有利于形成有效处方的标准,然后我们尝试了基于图卷积神经网络的方法,通过对药材配对的图结构进行学习,得到了一定的预测效果,但是由于数据量和模型复杂性的限制,泛化能力有待进一步提高。

\section{展望}

在未来的工作中,继续对中药处方、药材化合物含量、化合物对基因表达的影响、基因间的交互作用进行定量收集和测定将是有价值的工作,可为中药处方有效性预测的任务提供更多的数据支持,提高预测能力和可靠性。同时,我们也可以尝试更多的深度学习方法来进行更加符合数据特点的建模。在这些定量信息中,现阶段仍广泛缺失、亟待补充的是化合物对基因表达的影响,或即剂量-效应关系。找到足够规范、统一、可利用的数据表示方法将可能极大地提高从中药处方到基因互作网络关联关系的严谨性。除此之外,适当地引入更多的先验知识,如中药药理学、中药方剂学,以及现代药理学、药效学等领域的知识,将有助于提高模型的可解释性,构成一个更加完整的中药处方有效性预测框架。